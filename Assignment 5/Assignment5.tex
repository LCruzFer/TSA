% Options for packages loaded elsewhere
\PassOptionsToPackage{unicode}{hyperref}
\PassOptionsToPackage{hyphens}{url}
%
\documentclass[
]{article}
\usepackage{lmodern}
\usepackage{amssymb,amsmath}
\usepackage{ifxetex,ifluatex}
\ifnum 0\ifxetex 1\fi\ifluatex 1\fi=0 % if pdftex
  \usepackage[T1]{fontenc}
  \usepackage[utf8]{inputenc}
  \usepackage{textcomp} % provide euro and other symbols
\else % if luatex or xetex
  \usepackage{unicode-math}
  \defaultfontfeatures{Scale=MatchLowercase}
  \defaultfontfeatures[\rmfamily]{Ligatures=TeX,Scale=1}
\fi
% Use upquote if available, for straight quotes in verbatim environments
\IfFileExists{upquote.sty}{\usepackage{upquote}}{}
\IfFileExists{microtype.sty}{% use microtype if available
  \usepackage[]{microtype}
  \UseMicrotypeSet[protrusion]{basicmath} % disable protrusion for tt fonts
}{}
\makeatletter
\@ifundefined{KOMAClassName}{% if non-KOMA class
  \IfFileExists{parskip.sty}{%
    \usepackage{parskip}
  }{% else
    \setlength{\parindent}{0pt}
    \setlength{\parskip}{6pt plus 2pt minus 1pt}}
}{% if KOMA class
  \KOMAoptions{parskip=half}}
\makeatother
\usepackage{xcolor}
\IfFileExists{xurl.sty}{\usepackage{xurl}}{} % add URL line breaks if available
\IfFileExists{bookmark.sty}{\usepackage{bookmark}}{\usepackage{hyperref}}
\hypersetup{
  pdftitle={Time Series Assignment 5},
  pdfauthor={Lucas Cruz Fernandez},
  hidelinks,
  pdfcreator={LaTeX via pandoc}}
\urlstyle{same} % disable monospaced font for URLs
\usepackage[margin=1in]{geometry}
\usepackage{color}
\usepackage{fancyvrb}
\newcommand{\VerbBar}{|}
\newcommand{\VERB}{\Verb[commandchars=\\\{\}]}
\DefineVerbatimEnvironment{Highlighting}{Verbatim}{commandchars=\\\{\}}
% Add ',fontsize=\small' for more characters per line
\usepackage{framed}
\definecolor{shadecolor}{RGB}{248,248,248}
\newenvironment{Shaded}{\begin{snugshade}}{\end{snugshade}}
\newcommand{\AlertTok}[1]{\textcolor[rgb]{0.94,0.16,0.16}{#1}}
\newcommand{\AnnotationTok}[1]{\textcolor[rgb]{0.56,0.35,0.01}{\textbf{\textit{#1}}}}
\newcommand{\AttributeTok}[1]{\textcolor[rgb]{0.77,0.63,0.00}{#1}}
\newcommand{\BaseNTok}[1]{\textcolor[rgb]{0.00,0.00,0.81}{#1}}
\newcommand{\BuiltInTok}[1]{#1}
\newcommand{\CharTok}[1]{\textcolor[rgb]{0.31,0.60,0.02}{#1}}
\newcommand{\CommentTok}[1]{\textcolor[rgb]{0.56,0.35,0.01}{\textit{#1}}}
\newcommand{\CommentVarTok}[1]{\textcolor[rgb]{0.56,0.35,0.01}{\textbf{\textit{#1}}}}
\newcommand{\ConstantTok}[1]{\textcolor[rgb]{0.00,0.00,0.00}{#1}}
\newcommand{\ControlFlowTok}[1]{\textcolor[rgb]{0.13,0.29,0.53}{\textbf{#1}}}
\newcommand{\DataTypeTok}[1]{\textcolor[rgb]{0.13,0.29,0.53}{#1}}
\newcommand{\DecValTok}[1]{\textcolor[rgb]{0.00,0.00,0.81}{#1}}
\newcommand{\DocumentationTok}[1]{\textcolor[rgb]{0.56,0.35,0.01}{\textbf{\textit{#1}}}}
\newcommand{\ErrorTok}[1]{\textcolor[rgb]{0.64,0.00,0.00}{\textbf{#1}}}
\newcommand{\ExtensionTok}[1]{#1}
\newcommand{\FloatTok}[1]{\textcolor[rgb]{0.00,0.00,0.81}{#1}}
\newcommand{\FunctionTok}[1]{\textcolor[rgb]{0.00,0.00,0.00}{#1}}
\newcommand{\ImportTok}[1]{#1}
\newcommand{\InformationTok}[1]{\textcolor[rgb]{0.56,0.35,0.01}{\textbf{\textit{#1}}}}
\newcommand{\KeywordTok}[1]{\textcolor[rgb]{0.13,0.29,0.53}{\textbf{#1}}}
\newcommand{\NormalTok}[1]{#1}
\newcommand{\OperatorTok}[1]{\textcolor[rgb]{0.81,0.36,0.00}{\textbf{#1}}}
\newcommand{\OtherTok}[1]{\textcolor[rgb]{0.56,0.35,0.01}{#1}}
\newcommand{\PreprocessorTok}[1]{\textcolor[rgb]{0.56,0.35,0.01}{\textit{#1}}}
\newcommand{\RegionMarkerTok}[1]{#1}
\newcommand{\SpecialCharTok}[1]{\textcolor[rgb]{0.00,0.00,0.00}{#1}}
\newcommand{\SpecialStringTok}[1]{\textcolor[rgb]{0.31,0.60,0.02}{#1}}
\newcommand{\StringTok}[1]{\textcolor[rgb]{0.31,0.60,0.02}{#1}}
\newcommand{\VariableTok}[1]{\textcolor[rgb]{0.00,0.00,0.00}{#1}}
\newcommand{\VerbatimStringTok}[1]{\textcolor[rgb]{0.31,0.60,0.02}{#1}}
\newcommand{\WarningTok}[1]{\textcolor[rgb]{0.56,0.35,0.01}{\textbf{\textit{#1}}}}
\usepackage{graphicx,grffile}
\makeatletter
\def\maxwidth{\ifdim\Gin@nat@width>\linewidth\linewidth\else\Gin@nat@width\fi}
\def\maxheight{\ifdim\Gin@nat@height>\textheight\textheight\else\Gin@nat@height\fi}
\makeatother
% Scale images if necessary, so that they will not overflow the page
% margins by default, and it is still possible to overwrite the defaults
% using explicit options in \includegraphics[width, height, ...]{}
\setkeys{Gin}{width=\maxwidth,height=\maxheight,keepaspectratio}
% Set default figure placement to htbp
\makeatletter
\def\fps@figure{htbp}
\makeatother
\setlength{\emergencystretch}{3em} % prevent overfull lines
\providecommand{\tightlist}{%
  \setlength{\itemsep}{0pt}\setlength{\parskip}{0pt}}
\setcounter{secnumdepth}{-\maxdimen} % remove section numbering

\title{Time Series Assignment 5}
\author{Lucas Cruz Fernandez}
\date{}

\begin{document}
\maketitle

\#\#Question 9 We first read in the data and again calculate the growth
rate \(r_t\) as in Assignment 3. The code for this is omitted for
brevity.

We save the last 6 observations for calculating the Root Men Squared
Error at the end and define some general parameters that are of use
later.

\begin{Shaded}
\begin{Highlighting}[]
\NormalTok{no_models =}\StringTok{ }\DecValTok{4} \CommentTok{#number of models we want to estimate}
\NormalTok{n <-}\StringTok{ }\KeywordTok{length}\NormalTok{(growth_rate)  }\CommentTok{#helpful when "reducing" data sets later}
\NormalTok{valid <-}\StringTok{ }\NormalTok{growth_rate[}\KeywordTok{c}\NormalTok{((n}\DecValTok{-5}\NormalTok{)}\OperatorTok{:}\NormalTok{n)] }\CommentTok{#contains last 6 elements of growth_rate}
\end{Highlighting}
\end{Shaded}

The below function creates a one-step ahead prediction using the
supplied \(data\) and binds all predictions together in one list, which
is returned.

\begin{Shaded}
\begin{Highlighting}[]
\NormalTok{forecast_onestep <-}\StringTok{ }\ControlFlowTok{function}\NormalTok{(data, no_models)\{  }
\NormalTok{  prediction_list <-}\StringTok{ }\KeywordTok{vector}\NormalTok{()  }\CommentTok{#list of predictions to be returned}
  \ControlFlowTok{for}\NormalTok{ (i }\ControlFlowTok{in} \KeywordTok{seq}\NormalTok{(}\DataTypeTok{from=}\DecValTok{0}\NormalTok{, }\DataTypeTok{to=}\NormalTok{(no_models }\OperatorTok{-}\StringTok{ }\DecValTok{1}\NormalTok{)))\{     }\CommentTok{#loop over the models we are interested in }
\NormalTok{    model <-}\StringTok{ }\KeywordTok{arima}\NormalTok{(data, }\DataTypeTok{order =} \KeywordTok{c}\NormalTok{(i, }\DecValTok{0}\NormalTok{, }\DecValTok{0}\NormalTok{), }
                 \DataTypeTok{optim.control =} \KeywordTok{list}\NormalTok{(}\DataTypeTok{maxit =} \DecValTok{1000}\NormalTok{))    }\CommentTok{#estimate the model}
\NormalTok{    prediction <-}\StringTok{ }\KeywordTok{predict}\NormalTok{(model, }\DataTypeTok{n.ahead =} \DecValTok{1}\NormalTok{, }\DataTypeTok{se.fit=}\OtherTok{FALSE}\NormalTok{) }\CommentTok{#make prediction based on model}
\NormalTok{    prediction_list <-}\StringTok{ }\KeywordTok{c}\NormalTok{(prediction_list, prediction)   }\CommentTok{#add prediction to prediction list}
\NormalTok{  \}}
  \KeywordTok{return}\NormalTok{(prediction_list)}
\NormalTok{\}}
\end{Highlighting}
\end{Shaded}

We will use this function now to forecast the last 6 values of the
dataset. First, we create an empty \(4\times 6\) matrix, where the
columns represent the period we forecast and each row represents a
different model. The forecasting process in the \emph{for}-loop then
works as follows: to predict the last period we use all but the last
observation, thus, we remove it from the dataset. We use this
``reduced'' dataset to make the forecast using our before defined
function. The result of this function - the list of predictions from
each model - is then added as the \(i^th\) column of the matrix. The
second to last period is then predicted using all but the last two
observations and so on. In the end, the matrix contains all predictions
of all models for the periods of interest (e.g.~element {[}2, 2{]} of
the matrix is the prediction of the second to last period using an AR(1)
model).

\begin{Shaded}
\begin{Highlighting}[]
\CommentTok{#apply function to different datasets to get different predictions }
\NormalTok{n <-}\StringTok{ }\KeywordTok{length}\NormalTok{(growth_rate)}
\NormalTok{predictions_mat <-}\StringTok{ }\KeywordTok{matrix}\NormalTok{(}\DataTypeTok{data =} \OtherTok{NA}\NormalTok{, }\DataTypeTok{nrow =} \DecValTok{4}\NormalTok{, }\DataTypeTok{ncol =} \DecValTok{6}\NormalTok{)}
\ControlFlowTok{for}\NormalTok{(i }\ControlFlowTok{in} \DecValTok{1}\OperatorTok{:}\DecValTok{6}\NormalTok{)\{}
\NormalTok{  data <-}\StringTok{ }\NormalTok{growth_rate[}\KeywordTok{c}\NormalTok{(}\DecValTok{1}\OperatorTok{:}\NormalTok{(n}\OperatorTok{-}\NormalTok{i))]}
\NormalTok{  predictions_mat[, i] <-}\StringTok{ }\KeywordTok{forecast_onestep}\NormalTok{(}\DataTypeTok{data =}\NormalTok{ data, }\DataTypeTok{no_models =}\NormalTok{ no_models)}
\NormalTok{\}}
\NormalTok{predictions_mat <-}\StringTok{ }\KeywordTok{Rev}\NormalTok{(predictions_mat, }\DecValTok{2}\NormalTok{) }\CommentTok{#predictions returned are in "wrong" order; }
\CommentTok{#we want the first column to represent the eraliest forecasted period (T - 6), not the last}
\end{Highlighting}
\end{Shaded}

Finally, to calculate the RMSE we use a predefined function from the
Metrics package. The errors vector is filled with the corresponding RMSE
of each model.

\begin{Shaded}
\begin{Highlighting}[]
\NormalTok{errors <-}\StringTok{ }\KeywordTok{vector}\NormalTok{(}\DataTypeTok{length =}\NormalTok{ no_models)}
\ControlFlowTok{for}\NormalTok{(i }\ControlFlowTok{in} \DecValTok{1}\OperatorTok{:}\KeywordTok{dim}\NormalTok{(predictions_mat)[}\DecValTok{1}\NormalTok{])\{}
\NormalTok{  errors[i] <-}\StringTok{ }\KeywordTok{rmse}\NormalTok{(valid, predictions_mat[i, ])}
\NormalTok{\}}
\end{Highlighting}
\end{Shaded}

In this list of errors the second one is the smallest. The second model
we used to create the one-step ahead forecasts is an \(AR(1)\) model.
Thus, based on the RMSE this is the best model to forecast the GDP
growth rate.

\begin{Shaded}
\begin{Highlighting}[]
\KeywordTok{which}\NormalTok{(errors }\OperatorTok{==}\StringTok{ }\KeywordTok{min}\NormalTok{(errors))}
\end{Highlighting}
\end{Shaded}

\begin{verbatim}
## [1] 2
\end{verbatim}

\end{document}
